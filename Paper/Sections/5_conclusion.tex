
~\\~
\section{Conclusions} \label{sec:conclusions}
As the generalized trend in computing is to move towards manycore architectures and more etherogeneous machines, resource awareness is becoming more and more important in order to efficiently use the extremely high number of resources involved in a computation.
Extremely parallel environments are very sensitive to bottlenecks and inefficiencies and a dynamically adaptive and resource aware execution can help in avoiding performance degradation.

However, in order to exploit the potential of resource awareness there is also the need for a flexible and dynamic programming model: task-based parallelism has proven to be a powerful approach for increasing the efficiency of parallel applications and is available in several popular runtime environments.

HPX is a modern C++ runtime system for high-performance task-based parallelization. It offers an homogeneous interface for both shared memory and distributed memory parallelization and it can outperform both OpenMP and MPI in their respective fields of application. It has efficient and adaptive thread scheduling capabilities and it incorporates a performance monitoring framework which allows building resource-aware heuristics into application.

HPX also implements parallel algorithms according to the C++ standard, making high-performance high-level parallelism available also to non specialists.

Thanks to its performance, to its solid theoretical foundations and to its flexibility, HPX has the potential to become one of the major next generation parallelization paradigms. As MPI and OpenMP start showing their architectural limits, HPX should be considered as a valid alternative, not only for high performance computing but also for general purpose parallelism.

~\\~
