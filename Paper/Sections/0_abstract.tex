
\begin{abstract}
Resource awareness is the ability of a program to dinamically read the state of the resources it consumes and to adapt its execution accordingly.
This awareness can be exploited to achieve a more efficient resource usage, reduce overheads, improve performance and optimize energy consumption.
As HPC systems grow in size and complexity, resource awareness can also offer a way to reduce the effort required for tuning and portability.
As applications use runtime systems for parallelization, there is the need for runtime systems to support and expose resource awareness in an efficient and integrated way.
HPX is a modern C++ runtime system for high-performance task-based parallelization which is, to some degree, resource aware and adaptive by design.
It supports load-aware task scheduling and a powerful performance monitoring framework which can be used for runtime introspection and for building dynamic resource management heuristics.
Recent research efforts also show how HPX can support dynamic task grain size control and how HPX parallel algorithms can be dynamically tuned by using a mix of compile-time and runtime information for better performance.
\end{abstract}
