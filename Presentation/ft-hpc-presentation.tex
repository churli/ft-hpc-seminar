\documentclass[compress]{beamer}
\usepackage{lmodern} % Fixing beamer issues
\usepackage[T1]{fontenc}
%%% Support some german text
% \usepackage{ngerman}
% \usepackage[latin1]{inputenc}   % für Umlaute
%%%
\usepackage[utf8]{inputenc}
%%%
\usepackage[english]{babel}
\usepackage{amsmath}
\usepackage{amsfonts}
% \usepackage[inline]{enumitem} % To get roman numbers, breaks normal arrows...
\usepackage{microtype}
% \usepackage{eulervm}
% \usepackage{array} % needed for \arraybackslash
% \usepackage{graphicx}
% \usepackage{adjustbox} % for \adjincludegraphics
% \usepackage{soul}
% \usepackage{color}
% \usepackage{animate}

% \usepackage{../../TesiTriennale/Manoscritto/Styles/commands}

\usepackage{listings}

\newcommand{\nota}[1]{\textcolor{red}{#1}}

\title[]{How runtime systems can support resource awareness in HPC: the HPX case}
\author{Tommaso Bianucci}
\date{22 June 2018}
\institute{Technische Universität München}
%\logo{\includegraphics[width=15mm]{logo}}
\usetheme{Dresden}
% \usecolortheme{beaver}
%\useoutertheme[right]{sidebar}
\setbeamercovered{dynamic}
% \setbeamertemplate{theorems}[numbered]
% \theoremstyle{definition}
% \newtheorem{definizione}{Definizione}
% \theoremstyle{plain}
% \newtheorem{teorema}{Teorema}

% Mastruzzo per poter controllare gli odiosi pallini di navigazione
\makeatletter
\let\beamer@writeslidentry@miniframeson=\beamer@writeslidentry
\def\beamer@writeslidentry@miniframesoff{%
  \expandafter\beamer@ifempty\expandafter{\beamer@framestartpage}{}% does not happen normally
  {%else
    % removed \addtocontents commands
    \clearpage\beamer@notesactions%
  }
}
\newcommand*{\miniframeson}{\let\beamer@writeslidentry=\beamer@writeslidentry@miniframeson}
\newcommand*{\miniframesoff}{\let\beamer@writeslidentry=\beamer@writeslidentry@miniframesoff}
\makeatother
% 
\begin{document}

\begin{frame}
\maketitle
\end{frame}

\section{Introduction}
\begin{frame}
	\frametitle{Exascale will be hard}
	\begin{itemize}
		\item ExaFLOPS/s = $10^{18}$ FLOPS/s
		\item Billions of cores?
		% \item $10^4$ nodes
		\item Etherogeneous hardware
		\begin{itemize}
			\item Manycore CPUs
			\item GPUs
			\item FPGAs
		\end{itemize}
	\end{itemize}
	\pause
	\vspace{5mm}
	$\longrightarrow$ An extreme degree of parallelism is required.
\end{frame}

\begin{frame}
	\frametitle{Applications are hard}
	\begin{itemize}
		\item \emph{Scaling-impaired} applications
		% Esempio?
		\item Unbalanced execution tree
		% Qui esempio e' AMR
	\end{itemize}
	\pause
	This causes:
	\begin{itemize}
		\item Poor parallel performance
		\item Suboptimal resource usage
	\end{itemize}
	\pause
	\vspace{5mm}
	$\longrightarrow$ Some applications do not scale well.
	% e qui spiega cosa vuol dire scalare
\end{frame}

\begin{frame}
	\frametitle{Programming model matters}
	Current predominant model in HPC:
	\begin{itemize}
		\item \emph{Fork-join} for shared memory (OpenMP)
		\pause
		\item \emph{Communicating Sequential Processes} for distributed memory (MPI)
	\end{itemize}
	\pause
	Problems:
	\begin{itemize}
		\item Global barriers
		\item Load imbalance
	\end{itemize}
% We will see an alternative later
\end{frame}

\begin{frame}
	\frametitle{Programming model matters (2)}
	\vspace{-3mm}
	\begin{center}
		\includegraphics[width=80mm]{Figures/globalBarriersAndThreadIdleTime.png}\\
		\tiny P. A. Grubel: "Dynamic adaptation in hpx - a task-based parallel runtime system" 2016.
		\normalsize
	\end{center}
	\pause
	$\longrightarrow$ Limits of programming models can limit parallelism.
	% e qui spiega cosa vuol dire scalare
	% here it is also important to mention various overheads of parallelization!
\end{frame}

\section{Resource awareness}
\begin{frame}
	% \frametitle{Running blindly does not pay}
	\frametitle{Complex hardware requires clever software}
	\pause
	Resource awareness
	\vspace{5mm}
	\begin{itemize}
		\item Adaptive allocation and usage of resources
		\item The system is aware of its own resources
		\item At runtime vs. before execution
	\end{itemize}
\end{frame}

\begin{frame}
	\frametitle{What are resources?}
	\begin{enumerate}
		\item Hardware entities
			\begin{itemize}
				\item Computational units
				\item Memory
				\item Bus/network bandwidth
				\item I/O devices
				\item Power
				\item Thermal
			\end{itemize}
		\pause
		\item Software entities
			\begin{itemize}
				\item Buffers
				\item Queues
			\end{itemize}
	\end{enumerate}
\end{frame}

\begin{frame}
	\frametitle{Different levels of awareness}
	\begin{enumerate}
		\item Embedded computing
		% Cosa vuol dire di preciso embedded? Che ci scriviamo roba baremetal?
		\begin{itemize}
			\item Deal with power, aging, thermal effects/problems
			\item Manage low-level access to resources\\
			E.g.: Invasive computing on MPSoC
		\end{itemize}
		\pause
		\item Application/runtime system level
		\begin{itemize}
			\item Steer the execution to dynamically fit resources\\
			E.g.: load balancing, task scheduling
			\item Query resource statistics at runtime\\
			E.g.: task grain size tuning
		\end{itemize}
		\pause
		\item Supercomputing facility
		\begin{itemize}
			\item Efficient elastic job scheduling\\
			E.g.: Invasive MPI + job scheduler integration
		\end{itemize}
	\end{enumerate}
\end{frame}

\section{HPX}
\begin{frame}
	\frametitle{High Performance paralleX}
	% Here we dig into the 2 level of awareness
	\pause
	C++ runtime system for
	\vspace{3mm}
	\begin{itemize}
		\item \emph{Task-based} parallelism
		% definisci bene task-based
		\item \emph{Shared memory} + \emph{Distributed memory} parallelization
		\item Fine-grained parallelism
	\end{itemize}
\end{frame}

\begin{frame}
	\frametitle{HPX foundations}
	\begin{columns}
		\column{\dimexpr\linewidth-50mm-2mm}
		\begin{itemize}
			\item Asynchronous scheduling and execution
			% Instead of synchronous "parallel regions"
			\item Lightweight synchronization
			% Instead of barriers, just synch based on data dependencies
			\item Active Global Address Space (AGAS)
			% Direct addressing of remote resources instead of message passing semantics
			\item Performance monitoring framework
		\end{itemize}
		\column{50mm}
	 		\includegraphics[width=50mm]{Figures/hpxArchitecture.png}
	 		\tiny T. Heller et al.: “Hpx – an open source c++ standard library for parallelism and concurrency” 2017.
		\normalsize
	\end{columns}
\end{frame}

% \subsection{HPX 101}
\begin{frame}
	\frametitle{Futures}
	\begin{center}
		\includegraphics[width=100mm]{Figures/futureFlowDiagram.png}\\
		\tiny H. Kaiser et al.: “Parallex an advanced parallel
execution model for scaling-impaired applications” 2009.
		\normalsize
	\end{center}
\end{frame}

\begin{frame}
	\frametitle{Programming model}
	\begin{center}
		\includegraphics[width=100mm]{Figures/hpxProgrammingModel1.png}\\
		\vspace{3mm}
		\tiny T. Heller: “C++ on its way to exascale and beyond - The HPX Parallel Runtime System” 2016.
		\normalsize
	\end{center}
\end{frame}
\begin{frame}
	\frametitle{Programming model}
	\begin{center}
		\includegraphics[width=100mm]{Figures/hpxProgrammingModel2.png}\\
		\vspace{3mm}
		\tiny T. Heller: “C++ on its way to exascale and beyond - The HPX Parallel Runtime System” 2016.
		\normalsize
	\end{center}
\end{frame}
\begin{frame}
	\frametitle{Programming model}
	\begin{center}
		\includegraphics[width=100mm]{Figures/hpxProgrammingModel3.png}\\
		\vspace{3mm}
		\tiny T. Heller: “C++ on its way to exascale and beyond - The HPX Parallel Runtime System” 2016.
		\normalsize
	\end{center}
\end{frame}
\begin{frame}
	\frametitle{Programming model}
	\begin{center}
		\includegraphics[width=100mm]{Figures/hpxProgrammingModel4.png}\\
		\vspace{3mm}
		\tiny T. Heller: “C++ on its way to exascale and beyond - The HPX Parallel Runtime System” 2016.
		\normalsize
	\end{center}
\end{frame}

\section{RA in HPX}
\begin{frame}
	\frametitle{HPX and Resource Awareness}
	\pause
	Capabilities
	\begin{enumerate}
		\item Task scheduling\\
			\small $\longrightarrow$ Work stealing + NUMA-awareness
		\normalsize
		\pause
		\item AGAS\\
			\small $\longrightarrow$ Dynamic relocation of objects
		\normalsize
		\pause
		\item Percolation\\
			\small $\longrightarrow$ Directly addressing HW accelerators
		\normalsize
		\pause
		\item Performance counters\\
			\small $\longrightarrow$ Allow easier integration of awareness into applications
		\normalsize
	\end{enumerate}
\end{frame}

\begin{frame}
	\frametitle{HPX and Resource Awareness (2)}
	Limitations
	\begin{itemize}
		\item (An)elasticity of HPX\\
			\small $\longrightarrow$ Worker threads and localities cannot be changed at runtime
		\normalsize
		\pause
		\item Energy unawareness\\
			\small $\longrightarrow$ E.g. no DVFS support
		\normalsize
		\pause
		\item Fault tolerance\\
			\small $\longrightarrow$ No built-in facility
		\normalsize
	\end{itemize}
\end{frame}

\section{Example}
\begin{frame}
	\frametitle{HPX coding example: the Mandelbrot set}
	% \begin{itemize}
	% 	\item Foo
	% \end{itemize}
	\begin{columns}
		\column{\dimexpr\linewidth-65mm-2mm}
		\pause
		\begin{equation*}
			\begin{cases}
				z_{c,0} &= 0 \\
				z_{c,n+1} &= z_{c,n}^2 + c
			\end{cases}
		\end{equation*}
		\pause
		\begin{align*}
			\mathcal{M} & = \\
			& \{c\in\mathbb{C} : \lim_{n\to\infty} |z_{c,n}| < +\infty \}
		\end{align*}
		\column{65mm}
		\begin{center}
			\includegraphics[width=65mm]{Figures/mandelbrot.png}
		\end{center}
	\end{columns}
\end{frame}

\begin{frame}[fragile]
	\frametitle{Mandelbrot: kernel}
	TODO: here insert a listing of futurized code
\end{frame}

\begin{frame}[fragile]
	\frametitle{Mandelbrot: sequential code}
	\begin{lstlisting}[language=C]
		a = 20;
	\end{lstlisting}
\end{frame}

\begin{frame}[fragile]
	\frametitle{Mandelbrot: futurized code}
	TODO: here insert a listing of futurized code
\end{frame}

\begin{frame}
	\frametitle{Mandelbrot step by step}
	\begin{center}
		% \animategraphics[loop,autoplay,scale=0.15]{10}{Figures/AnimatedMandelbrot/mandelbrot_}{00000}{000099}
		\includegraphics[width=65mm]{Figures/mandelbrot_partial.png}
	\end{center}
\end{frame}

\section{Conclusions}
\begin{frame}
	\frametitle{Conclusions}
	\begin{itemize}
		\item Future exascale computing requires smart code.
		\item Resource awareness can be a way to achieve better performance.
		\item HPX has the potential to become a major runtime system for HPC, thanks to both its performance and programmability.
	\end{itemize}
	\pause
	\vspace{5mm}
	\begin{center}
		\LARGE Thanks! \normalsize
	\end{center}
\end{frame}

%%%%%%%

% \begin{frame}
% 	\frametitle{Title}
% 	\begin{itemize}
% 		\item Foo
% 	\end{itemize}
% \end{frame}

\end{document}
